\documentclass[11pt]{article}

\usepackage{listings}
\usepackage{epsfig}
\usepackage{lscape}
\usepackage{multirow}
\usepackage{longtable}
\usepackage{amsmath,amssymb,amsthm}
\usepackage{graphicx}
\usepackage{color}
\usepackage{placeins}
\usepackage{url}
\usepackage{cases}
\usepackage{hyperref}
\usepackage{setspace}
\usepackage{subfigure}
\usepackage{float}
\usepackage{listings}
\usepackage{framed}
\usepackage[c2]{optidef}
\usepackage{tikz}
\usepackage{caption}
\usepackage{algorithm}
\usepackage{algorithmic}


\usepackage{booktabs}

\oddsidemargin 0pt
\evensidemargin 0pt
\marginparwidth 10pt
\marginparsep 10pt
\topmargin -20pt
\textwidth 6.5in
\textheight 8.5in
\parindent = 20pt

\DeclareMathOperator*{\argmin}{argmin}
\DeclareMathOperator*{\minimax}{minimax}

\renewcommand{\algorithmicrequire}{ \textbf{function:}}
\renewcommand{\algorithmicreturn}{ \textbf{end function}}
\newcommand{\blue}[1]{\begin{color}{blue}#1\end{color}}
\newcommand{\magenta}[1]{\begin{color}{magenta}#1\end{color}}
\newcommand{\red}[1]{\begin{color}{red}#1\end{color}}
\newcommand{\green}[1]{\begin{color}{green}#1\end{color}}

\newtheorem{theorem}{Theorem}
\newtheorem{proposition}{Proposition}
\newtheorem{lemma}{Lemma}
\newtheorem{corollary}{Corollary}
\newtheorem{remark}{Remark}
\newtheorem{assumption}{Assumption}
\newtheorem{definition}{Definition}
%\newenvironment{proof}{{\noindent\it Proof}\quad}{\hfill $\square$\par}

%\usepackage{sidecap}


\begin{document}
	
	\title{\bf Simplex Report}
	
	\author{Boyuan Yao 19307110202}
	
	
	\date{\today}
	\maketitle

	
\section{Primal Simplex}
	
	My primal simplex method contains three stages: presolve, phaseI and phaseII. The first part is to do a simple processing of the original problem, in my implementation it involves two operations: delete some rows to make the matrix has full row rank(through QR decomposition) and eliminate the rows that only has one nonzero entry(which means that the corresponding decision variable is stationary). The second part is to find the initial BFS while the second one is for the optimal solution. The followings are the outcomes with the given testing instances.
	\begin{center}
		\scalebox{0.85}{\begin{tabular}{|c|c|c|c|c|c|c|c|c|}
				\hline
				& optvalue & tpresolve(s) & tphaseI(s) & tphaseII(s) & iter1 & iter2 & error\_solution & error\_outcome\\
				\hline
				25fv47 & 5.5018e+03 & 0.1762 & 267.8798 & 301.2520 & 98894 & 107553 & 8.6058e-16 & 1.3225e-15\\
				\hline
				brandy & 1.5185e+03 & 0.0120 & 0.2934 & 0.1382 & 1767 & 793 & 1.5217e-15 & 1.4973e-16\\
				\hline
				cre\_a & 2.3595e+07 &12.0023 & 1.5075e+03 & 1.4276e+03 & 16011 & 14180 & 3.5187e-15 & 3.5176e-16\\
				\hline
				cre\_d & 2.4455e+07 & 852.3142 & 6.4671e+03 & 3.7481e+04 & 10700 & 67499 & 6.5879e-15 & 7.4643e-15\\
				\hline
				e226 & -18.7519 & 0.0092 & 0.4611 & 0.8002 & 1235 & 2075 & 1.2793e-15 & 2.0840e-15\\
				\hline
				scrs8 & 904.2970 & 0.0568 & 2.5736 & 1.8096 & 2091 & 1474 & 1.1714e-16 & 0\\
				\hline
		\end{tabular}}
	\end{center}
	
	Optvalue means the optimal value attained by my primal simplex code, tpresolve, tphaseI and tphaseII mean the time spends in the corresponding stages. iter1 and iter2 mean the number of iterations in phaeI and phaseII. error\_solution means the relative error of the solution, which is calculated by $\frac{||Ax - b||_2}{||x||_2}$. error\_outcome compares the optimal value attained by my implementation with the one attained by MATLAB standard function linprog, which is calculated by $\frac{|optvalue - f^*|}{|f^*|}$.
	
	When testing nug08, I met degeneracy problem that phaseI failed to complete.
\section{Dual simplex}

	My dual simplex method only contains one stage, assuming that the initial basis is given, and the matrix A has full row rank. I use randomize cases to test the function, and compare it with my primal simplex:
	\begin{center}
		\scalebox{0.9}{\begin{tabular}{|c|c|c|c|c|c|c|c|c|c|}
				\hline
				& primal & tphaseII(s) & iter2 & dual & optvalue & t & iter & error\_solution & error\_outcome\\
				\hline
				case1 & & 0.7120 & 6611 & & 9.2184 & 0.0519 & 632 & 3.2434e-16 & 1.9270e-16\\
				\hline
				case2 & & 395.9219 & 317177 & & 31.0542 & 2.3556 & 2231 & 2.9422e-16 & 1.1440e-16\\
				\hline
				case3 & & 7.2580 & 876 & & 26.5685 & 2.7413 & 174 & 3.1100e-17 & 4.0116e-16\\
				\hline
		\end{tabular}}
	\end{center}
	
	Three testing instances are generating by my code test\_instance. Matrix has the structure $A = (\tilde{A},I)$. Here are some information about the three instances:
	\begin{center}
		\scalebox{1}{\begin{tabular}{|c|c|c|c|}
				\hline
				& row & column & density\\
				\hline
				case1 & 100 & 300 & 0.1\\
				\hline
				case2 & 500 & 1500 & 0.01\\
				\hline
				case3 & 1000 & 2000 & 0.002\\
				\hline
		\end{tabular}}
	\end{center}

	You could find all the codes in the folder "src", all the testing outcomes in the folder "outcome".
\end{document}


